\section{Sistemas Lineares}
\label{sec:sis}
Um sistema linear é qualquer agrupamento de equações com variáveis de expoente 1, como por exemplo:
\begin{displaymath}
\left\{
\begin{array}{l}
  3x+y=1\\2x+5y=3
\end{array}
\right.
\end{displaymath}
Um sistema pode ter, a principio, qualquer número de variáveis e equações, como veremos mais a frente.
Um sistema linear pode ser expresso de forma matricial. Usando o sistema do exemplo anterior, podemos re-escreve-lo como:
\[
\begin{bmatrix}
  3 & 1\\2 & 5 \end{bmatrix}
\cdot
\begin{bmatrix}
  x\\y \end{bmatrix}
=
\begin{bmatrix}
  1\\3 \end{bmatrix}
  \]
  Todo sistema linear pode ser escrito como $A\cdot X=B$, onde A é chamada de \textit{matriz dos coeficientes}, X é chamada de \textit{matriz das incógnitas} e B é chamada de \textit{matriz dos termos independentes}.
  
  Em sistemas se podem aplicar operações elementares às equações do sistema, de forma idêntica ao que se pode fazer com matrizes e o sistema resultante será um sistema equivalente, com as mesmas propriedades da equivalência de matrizes.
  \[\text{O sistema:}\left\{
  \begin{array}{l}
    3x+2y=4\\ 5x+y=3
  \end{array}\right.
  \sim
  \left\{
  \begin{array}{l}
    3x+2y=4\\ 2x-y=-1
  \end{array}\right.\]
  Onde subtrai a $2^a$ equação da $1^a$ e o sinal $\sim$ representa a relação de equivalência.
  \subsection{Resolução de Sistemas Lineares}
\subsubsection*{Método da substituição}
  \addcontentsline{toc}{subsubsection}{Método da substituição}
  Resolver um sistema linear significa determinar o valor das incógnitas de forma que todas as equações do sistema sejam simultaneamente verdadeiras ou verificar que isso é impossível, como será visto mais adiante.\\
  Vamos resolver o sistema a seguir:
  \[
  \left\lbrace
  \begin{array}{lr}
  3x+y=1 & (1)\\
  x+y=3 & (2)  
  \end{array}
  \right.
  \]
Da equação (2), temos que y=3-x (3). Substituindo (3) na equação (1), resulta:
\[\begin{array}{l}
3x+y=1\\
3x+3-x=1\\
2x=-2 \\
x=-1\\
\text{Usando (3) descobrimos que y=4}
\end{array}
\]
Esse é o método da substituição. Para mais de 2 variáveis o processo é similar, no entanto vai resultar em contas mais complicadas. Nesses casos, com mais de 2 variáveis, é melhor usar o método de Gauss, explicado mais a frente.
\paragraph*{Substituição retroativa}
Quando se tem um sistema escalonado, como:
\begin{equation*}
\left\lbrace
\begin{array}{l}
ax+by+cz=l\\
\qquad dy+ez=m\\
\qquad \qquad fz=n
\end{array}
\right.
\end{equation*}
Então as soluções serão:
\begin{equation*}
  \begin{array}{l}
    z=\frac{\displaystyle n}{\displaystyle f}\\
    \text{Conhecendo z, achamos y:}\\
    y=\frac{\displaystyle m-ez}{\displaystyle d}\\
    \text{Sabendo z e y, x resulta:}\\
    x=\frac{\displaystyle e-cz-by}{\displaystyle a}
  \end{array}
\end{equation*}
Esse algoritmo de encontrar as incógnitas por substituições sucessivas se chama \textit{substituição retroativa}.
  \subsubsection*{Método de Gauss}
  \addcontentsline{toc}{subsubsection}{Método de Gauss}
  O método de Gauss é um algoritmo para fazer uma matriz (ou um sistema) adquirir a forma triangular superior. Ele usa \textit{pivôs} na diagonal principal. A cada operação se deve zerar todos os elementos abaixo do pivô, e para isso se deve multiplicar a linha do pivô por um número conveniente (o multiplicador) e adiciona-la à linha que se quer zerar os termos. Para encontrar o multiplicador você deve dividir o número que você quer zerar pelo pivô e usar o sinal inverso.\\
  Por exemplo na matriz $\left(\begin{smallmatrix}2&4\\1&0\end{smallmatrix}\right)$ você tem como $1^o$ pivô o 2 e o multiplicador da $1^a$ linha será -(1/2). A seguir vamos fazer um exemplo completo:
  \[\left\{
    \begin{array}{l}
    x+2y+z=2\\
    \qquad 3y+2z=1\\
    5x+y+z=5
    \end{array}\right.\]
    O $ 1^{\underline{o}} $ pivô é o coeficiente de x na $1^a$ equação. O multiplicador da $3^a$ linha é $-5/1=-5$.
Então, multiplicamos a $1^a$ linha por -5 e somamos na última, resultando:
    \[\left\{
    \begin{array}{l}
    x+2y+z=2\\
    \qquad 3y+2z=1\\
    \quad -9y-4z =--5  
    \end{array}\right.\]
    Agora o pivô é o coeficiente de y na $2^a$ equação, 3, e o multiplicador da $3^a$ linha é $-(-9)/3=3$.
Multiplicando a $2^a$ linha por 3 e somando na $3^a$, temos:
    \[\left\{
    \begin{array}{l}
    x+2y+z=2\\
    \qquad 3y+2z=1\\
    \qquad\quad 2z =-2  
    \end{array}\right.\]
    Fazendo a substituição retroativa temos como solução $z=1$, $y= -\frac{1}{3}$ e $x=\frac{1}{3}$.
    
  \subsection{Discussão de Sistemas Lineares}
  Um sistema linear pode ser:
  \begin{description}
    \item[Possível e Determinado]É quando um sistema admite 1 única solução, como foi o caso do exemplo visto na seção
 anterior.
    \item[Possível e Indeterminado]É quando um sistema admite infinitas soluções.
    \item [Impossível]É quando o sistema não admite solução. Por exemplo:\\$\left\{\begin{array}{l}
      x+y=1\\x+y=2
    \end{array}\right.$ É um sistema sem solução.
  \end{description}
\paragraph{Sistema Possível e Indeterminado}
  Este tipo de sistema é reconhecido quando na forma escalonada ele apresenta uma linha do tipo $0x_1+0x_2+
\dots+0x_n=0$, onde $x_i$ são as $n$ variáveis do sistema. A notação de chamar as variáveis por $x_1, x_2, \cdots, x_n$ é 
mais versátil, às vezes, do que $x, y, z, ...$. A solução desse sistema é escrita em função de uma
 ou mais variáveis, que são chamadas de \textit{variáveis livres}. Por exemplo o sistema:\[
  \left\{\begin{array}{l}
    4x+y+z=2\\2x+5y+3z=3\\6x+6y+4z=5
  \end{array}\right.\]
  Na forma escalonada fica:\[
  \left\{\begin{array}{l}
    4x+y+z=2\\0x+4,5y+2,5z=2\\0x+0y+0z=0
  \end{array}\right.\]
  Selecionando z como a variável livre, podemos escrever x e y em função de z, resultando como conjunto solução:
    \[
      S=\left\{ \left( x=\frac{7}{18}-\frac{1}{9}z,y=\frac{4}{9}-\frac{5}{9}z\right)\right\}
    \]
    Você percebe que existe um par de soluções para cada valor de z, e existem infinitos valores de z possíveis, logo existem infinitas soluções.
    \paragraph{Sistemas Impossíveis}
    Um sistema é impossível quando uma das linhas do sistema escalonado é do tipo: $0x_1+0x_2+\dots+0x_n=\beta$ onde $\beta$ é um número real.\[
    \text{Seja o sistema:}
    \left\{\begin{array}{l}
      x+y=2\\x+y=3
    \end{array}\right.
    \sim
    \left\{\begin{array}{l}
      x+y=2\\0x+0y=1
   \end{array}\right.\]
   Você percebe claramente que como $0\neq 1$ para quaisquer valores de x e y, então não existe solução para esse sistema.
