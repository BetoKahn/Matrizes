\section{Operações com matrizes}
\subsection{Multiplicação por Real}
Para multiplicar uma matriz A por um numero real $\alpha$, se deve multiplicar todos os elementos da matriz por $\alpha$, como a seguir:
\begin{displaymath}
\text{Seja a matriz A=}
\begin{bmatrix}
  a_{11} & a_{12}\\ a_{21} & a_{22}\\ a_{31} & a_{32}\\
\end{bmatrix}
\end{displaymath}
e seja $\alpha$ um número real qualquer, então:
\begin{displaymath}
\alpha \cdot A = \begin{bmatrix}
  \alpha \cdot a_{11} &\alpha \cdot  a_{12}\\ \alpha \cdot a_{21} &\alpha \cdot  a_{22}\\
  \alpha \cdot a_{31} & \alpha \cdot a_{32} \end{bmatrix}
\end{displaymath}
Em relação a multiplicação de uma matriz por um número real, vale a propriedade comutativa: \[A \cdot \alpha = \alpha \cdot A \]
Exemplo: Seja $ A= \left( \begin{smallmatrix} 11 & 26 \\ \pi & 5
\end{smallmatrix} \right)$. Calcular a matriz B, tal que $B=3A$.
%\begin{align*}
\begin{displaymath}
B=3A=3 \cdot
  \begin{bmatrix}
    11 & 26 \\ \pi & 5
  \end{bmatrix}
  =
  \begin{bmatrix}
    3 \cdot 11 & 3 \cdot 26 \\ 3 \cdot \pi & 3 \cdot 5
  \end{bmatrix}
  =
  \begin{bmatrix}
    33 & 78 \\ 3\pi & 15
  \end{bmatrix}
\end{displaymath}
%\end{align*}
\subsection{Soma e subtração}
A soma (e subtração) de matrizes se faz assim: Basta somar ( ou subtrair ) um a um os elementos das duas matrizes.
Sejam A e B duas matrizes e C = A + B, resulta $c_{ij}=a_{ij}+b_{ij}$.\\
Para simplicidade vou usar matrizes $2 \times 2$, mas muitas dessas operações valem para matrizes $n \times m$. Para matrizes genéricas $2 \times 2$ fica:
\begin{displaymath}
  A+B=
  \begin{bmatrix}
  a_{11} & a_{12}\\a_{21} & a_{22}  
  \end{bmatrix}
  +
  \begin{bmatrix}
  b_{11} & b_{12}\\b_{21} & b_{22}  
  \end{bmatrix}
  =
  \begin{bmatrix}
  a_{11}+b_{11} & a_{12}+b_{12}\\a_{21}+b_{21} & a_{22}+b_{22}  
  \end{bmatrix}
\end{displaymath}
Exemplo:
\begin{displaymath}
A=
\begin{bmatrix}
  1 & 2 \\ 3 & 4
\end{bmatrix}
 \quad B=\begin{bmatrix}
 5 & 6 \\ 7 & 8 \end{bmatrix}
\quad A + B =
\begin{bmatrix}
  1+5=6 & 2+6=8\\
  3+7=10 & 4+8=12
  \end{bmatrix}
\end{displaymath}
Fazendo agora a subtração $ B-A $, temos:
\begin{displaymath}
\begin{bmatrix}
5 & 6 \\
7 & 8
\end{bmatrix}
-
\begin{bmatrix}
1 & 2 \\ 
3 & 4
\end{bmatrix}
=
\begin{bmatrix}
5-1 & 6-2 \\ 
7-3 & 8-4
\end{bmatrix}
=
\begin{bmatrix}
4 & 4 \\ 
4 & 4
\end{bmatrix} 
\end{displaymath}
Para a adição de matrizes valem as propriedades:
\begin{description}
\item[Comutativa] $A+B=B+A$
\item[Associativa]$ A+(B+C)=(A+B)+C $
\end{description}
\subsection{Multiplicação}
A multiplicação de matrizes é um pouco mais complicada. Primeiro: O número de colunas da primeira matriz deve ser igual ao número de linhas da segunda, sem essa condição não é possível a multiplicação de matrizes e com alguma prática, você entenderá porque. 
Suponha duas matrizes quadradas 2x2 A e B, a multiplicação A$\cdot$B=P, resulta: 
\begin{displaymath}
\begin{bmatrix}
  a_{11} & a_{12}\\ a_{21} & a_{22}\\
\end{bmatrix}
\cdot
\begin{bmatrix}
  b_{11} & b_{12}\\ b_{21} & b_{22}\\
  \end{bmatrix}
=
\begin{bmatrix}
  a_{11}b_{11}+a_{12}b_{21} & a_{11}b_{12}+a_{12}b_{22} \\
  a_{21}b_{11}+a_{22}b_{21} & a_{21}b_{12}+a_{22}b_{22} \\
\end{bmatrix}
 =
\begin{bmatrix}
   p_{11} & p_{12} \\ p_{21} & p_{22} \\
\end{bmatrix}
\end{displaymath}
Agora você viu que o termo $p_{ij}$ se calcula a partir da linha i da matriz A e da coluna j da matriz B, se deve multiplicar termo a termo e somar os produtos dos elementos dessas linhas e colunas. Observe que a multiplicação de matrizes não é necessariamente comutativa, ou seja, $A\cdot B$ pode ser diferente de $B\cdot A$. Não existe divisão de matrizes.\\
Exemplo:\\
Sejam as matrizes:
\begin{displaymath}
  A=
  \begin{bmatrix}
    1 & 3 \\ 5 & 7
  \end{bmatrix}
  \text{ e }
  B=
  \begin{bmatrix}
    2 & 4 \\ 6 & 8
  \end{bmatrix}
\end{displaymath}
O produto $A \cdot B$ é:
\begin{displaymath}
A\cdot B=
\begin{bmatrix}
  1 \cdot 2+3 \cdot 6 & 1 \cdot 4+3 \cdot 8 \\
  5 \cdot 2+7 \cdot 6 & 5 \cdot 4+7 \cdot 8
\end{bmatrix}
=
\begin{bmatrix}
  20 & 28\\
  52 & 76
\end{bmatrix}
\end{displaymath}
O produto $B\cdot A$ resulta:
\begin{displaymath}
  B\cdot A=
  \begin{bmatrix}
    22 & 34\\46 & 74
  \end{bmatrix}
\end{displaymath}
Como foi falado antes você vê que $A\cdot B\neq B\cdot A$.
