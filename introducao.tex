\section{Introdução}
As matrizes são agrupamentos de números reais ou complexos organizados de forma ordenada em linhas e colunas e delimitados por grandes colchetes [] ou parenteses() . São objetos matemáticos com grande importância em física e matemática, como por exemplo, na matemática vetorial ou em problemas de mecânica dos corpos rígidos e em mecânica quântica e relatividade. As matrizes possuem propriedades importantes de se conhecer sobre as quais falarei neste trabalho.\\
Algumas definições de notação:
\begin{itemize}
  \item As matrizes são denotadas por letras maiúsculas. Ex. A letra \textbf{A} representa a matriz A.
  \item O número de linhas e colunas de uma matriz é escrito em subscrito junto ao nome da matriz. Exemplo: $A_{2\times3}$ significa que a matriz A possui 2 linhas e 3 colunas.
  \item Um elemento de uma matriz é denotado pela letra da matriz em forma minuscula com o número da linha e coluna (nessa ordem, sempre) em sobrescrito. Por exemplo, $a_{11}$ é o elemento da linha 1 e coluna 1 da matriz A.
  \item De forma mais genérica denotamos $a_{ij}$ o elemento da linha i e coluna j da matriz A.
\end{itemize}  

\paragraph*{Matriz retangular} é o tipo mais genérico de matriz, que possui número de linhas diferente do número de colunas. Um exemplo de matriz retangular:
\begin{displaymath}
\begin{bmatrix}
  1 & 2 & 3 \\  4 & 5 & 6
\end{bmatrix}.
\end{displaymath}

\paragraph*{Matriz quadrada} é uma matriz que possui o mesmo número de linhas e colunas, esse número é chamado de \textit{ordem} da matriz. A matriz $B_{2\times2}$ é dita de ordem 2 e pode ser escrita como $B_{2}$.Um exemplo de matriz quadrada de ordem 2:
\begin{displaymath}
\begin{bmatrix}
  2 & 4 \\ \pi & sen(30^0)
\end{bmatrix}
\end{displaymath}
\paragraph*{A Diagonal principal} de uma matriz quadrada A de ordem m $A_m$ são os elementos $a_{ij}$ tais que i=j ($a_{11},\, a_{22},\, a_{33},...$. Na matriz $4 \times 4$ a seguir a diagonal principal contem o símbolo **:
\begin{displaymath}
\begin{pmatrix}
** & 2 & 3 & 7\\
1 & ** & 34 & 3\\
34 & 6 & ** & 12\\
8,7 & 1 & 1 & ** \\
\end{pmatrix}
\end{displaymath}
\paragraph*{A Diagonal secundaria} da matriz $A_m$ é a diagonal que começa em $ a_{1m} $ e termina em $ a_{m1} $, indicada na matriz $ 3 \times 3 $ a seguir por **:
\begin{displaymath}
\begin{pmatrix}
1 & 2 & **\\
0 & ** & 5 \\
** & 4 & 7
\end{pmatrix}
\end{displaymath}
\paragraph{A Matriz transposta} de uma matriz $A_{m\times n}$ é a matriz $(A^t)_{n\times m}$, tal que
o elemento $(a^t)_{ij}$ da transposta é o elemento $(a_{ji})$ da matriz original, ou seja, as linhas de A viram as colunas da transposta.
Para uma matriz $A_{2 \times 3}$ genérica:
\begin{displaymath}
  A=
  \begin{bmatrix}
  a_{11} & a_{12} & a_{13} \\ 
  a_{21} & a_{22} & a_{23}
  \end{bmatrix}
  \; A^t=
  \begin{bmatrix}
  a_{11} & a_{21} \\ 
  a_{12} & a_{22} \\
  a_{13} & a_{23}
  \end{bmatrix}
\end{displaymath}
Um exemplo prático, a transposta da matriz $A_{2 \times 2}$ dada a seguir:
\[ A=
\begin{bmatrix}
1 & 2 \\ 3 & 4 \\
\end{bmatrix}
 \; A^t=
 \begin{bmatrix}
   1 & 3 \\ 2 & 4 \\
 \end{bmatrix}
 \]
